\section{Introduction}

Most modern CPUs have hardware performance counters; these
counters allow detailed, low-level measurement of processor behavior.
The counters are most commonly used for performance analysis, especially
in the High Performance Computing (HPC) field.  Usage has spread
to the desktop and embedded areas, with many new and novel
utilization scenarios.

There are a wide variety of events that can be measured with performance
counters, with event availability varying considerably among CPUs and vendors.
Some processors provide hundreds of events; separating the
useful and accurate ones from those that are broken and/or measure
esoteric architectural minutia can be a harrowing process.
Event details are buried in architectural manuals, often accompanied
by disclaimers disavowing any guarantees of useful results.

Counter validation is a difficult process.  Some events cannot
be validated effectively, as they require exact knowledge of
the underlying CPU architecture and can be influenced
by outside timing effects not under user control
~\cite{mytkowicz+:asplos09}.
This includes most cache events, cycle counts, and any event
affected by speculative execution. 

A subset of events exists that {\em is} architecturally specified;
these measure various kinds of retired instructions.
With a deterministic program (one that when provided
with the same input traverses the exact same code path
and generates the exact same output)
the counter results should be the same for every run.
These counts should be consistent; otherwise
the processor would not be ISA compatible with others in
the same architecture.

\subsection{The Need for Deterministic Events}

There are many situations where deterministic software execution
is necessary.  
Deterministic execution is useful when validating 
architectural simulators~\cite{weaver+:wddd08,desikan+:trut01},
when analyzing program behavior using 
basic block vectors (BBVs)~\cite{weaver+:hipeac08},
when performing Feedback Directed Optimization (FDO)~\cite{chen+:cgo10},
when using hardware checkpointing and 
rollback recovery~\cite{stodden+:isas06}, 
when performing intrusion analysis~\cite{dunlap+:osdi02},
and when implementing parallel deterministic 
execution.

Parallel deterministic execution enables debugging and analysis of 
multi-threaded applications in a repeatable way.  
Deterministic lock interleaving makes it possible to track down 
locking problems in large parallel applications.
There have been many proposals about how to best implement parallel
deterministic execution; many require modified hardware or modified
operating systems.  A quick and easy way to build deterministic
locks is to use hardware performance counters to ensure that
previously non-deterministic lock behavior happens
in a consistent, repeatable 
way~\cite{olszewski+:asplos09,yun:code2010,aviram+:osdi10,bergan+:osdi10}.
The need for parallel deterministic execution has been the primary
impetus for the search for deterministic performance events.

\subsection{Definitions}

In this work we search for {\em useful deterministic events}.
We define a useful deterministic event as one where the value
does not change run-to-run due to the microarchitecture of the 
processor (it is not affected by speculative execution),
the expected value can be determined via code inspection, and
the event occurs with enough frequency and distribution 
to be useful in program analysis.

We find two primary causes for events to deviate from the expected
result:
{\em nondeterminism} (identical runs returning different values) 
and {\em overcount} (some instructions counting multiple times).  
We investigate both sources of deviation.
